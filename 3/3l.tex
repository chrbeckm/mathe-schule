\documentclass[
  bibliography=totoc,
  captions=tableheading,
  titlepage=firstiscover,
]{scrartcl}
\usepackage{scrhack}
\usepackage[aux]{rerunfilecheck}
\usepackage{amsmath}
\usepackage{amssymb}
\usepackage{mathtools}
\usepackage{fontspec}
\recalctypearea{}
\usepackage{polyglossia}
\setmainlanguage{german}
\usepackage[
  math-style=ISO,
  bold-style=ISO,
  sans-style=italic,
  nabla=upright,
  partial=upright,
  warnings-off={
    mathtools-colon,
    mathtools-overbracket,
  },
]{unicode-math}
\setmathfont{Latin Modern Math}
\setmathfont{XITS Math}[range={scr, bfscr}]
\setmathfont{XITS Math}[range={cal, bfcal}, StylisticSet=1]
\usepackage[
  locale=DE,
  separate-uncertainty=true,
  per-mode=symbol-or-fraction,
]{siunitx}
\sisetup{math-micro=\text{µ},text-micro=µ}
\usepackage[
  version=4,
  math-greek=default,
  text-greek=default,
]{mhchem}
\usepackage[autostyle]{csquotes}
\usepackage{xfrac}
\usepackage{float}
\floatplacement{figure}{htbp}
\floatplacement{table}{htbp}
\RequirePackage{luatex85}
\usepackage[
  locale=DE,
]{siunitx}
\usepackage{tikz}
\usepackage[
  europeanresistors,
  americaninductors,
  siunitx,
]{circuitikz}
\usepackage{stackengine}
\AtBeginDocument{
  \sisetup{
    math-rm=\mathrm,
    math-micro=µ,
  }
}
\usepackage[
  section,
  below,
]{placeins}
\usepackage{pdflscape}
\usepackage[
  labelfont=bf,
  font=small,
  width=0.9\textwidth,
]{caption}
\usepackage{subcaption}
\usepackage{graphicx}
% \usepackage{grffile} % currently broken
\usepackage{booktabs}
\usepackage{microtype}
\usepackage[
  backend=biber,
]{biblatex}
\addbibresource{lit.bib}
\addbibresource{programme.bib}
\usepackage[
  unicode,
  pdfusetitle,
  pdfcreator={},
  pdfproducer={},
]{hyperref}
\usepackage{bookmark}
\usepackage[shortcuts]{extdash}
\usepackage{multirow}
\usepackage{rotating}
\setlength\parindent{0pt}

\begin{document}

\section*{Lineare Funktionen / Geraden - Lösung}

Aufgaben: Bestimme die
\begin{enumerate}
      \item Y-Achsenschnittpunkte,
      \item X-Achsenschnittpunkte,
      \item Schnittpunkte der Geraden untereinander.
\end{enumerate}
Fertige eine Zeichnung mit allen errechneten Punkten an.\\
Extra: Berechne die Fläche zwischen Koordinatenachsen und der jeweiligen Geraden.
\\
$f(x)=-5x+1 \;\;\;\; g(x)=-\frac{3}{5}x+3 \;\;\;\; h(x)=2x-2$

\begin{figure}
      \centering
      \includegraphics{build/plot.pdf}
      \caption{Zeichnung der Geraden mit allen errechneten Punkten.}
\end{figure}

\newpage
\section*{f(x)}
Y-Achsenschnittpunkt: $S_y (0\,|\,1)$
\begin{equation}
      f(0)=-5\cdot 0+1=1
\end{equation}
X-Achsenschnittpunkt: $S_x (0.2\,|\,0)$
\begin{align}
      0&=-5x+1\;\;\;\;|\!+\!5x\\
      5x&=1\;\;\;\;\;\;\;\;\;\;\;\;\;\;\;|\!:\!5\\
      x&=\frac{1}{5}=0.2
\end{align}
Flächeninhalt:
\begin{equation}
      \symup{A}_\text{f} = \frac{1}{2}\cdot1\cdot\frac{1}{5}=\frac{1}{10}=0.1
\end{equation}

\section*{g(x)}
Y-Achsenschnittpunkt: $S_y (0\,|\,3)$
\begin{equation}
      g(0)=-\frac{3}{5}\cdot 0+3=3
\end{equation}
X-Achsenschnittpunkt: $S_x (5\,|\,0)$
\begin{align}
      0&=-\frac{3}{5}x+3\;\;\;\;|\!+\!\frac{3}{5}x\\
      \frac{3}{5}x&=3\;\;\;\;\;\;\;\;\;\;\;\;\;\;\;\;|\!\cdot\!\frac{5}{3}\\
      x&=5
\end{align}
Flächeninhalt:
\begin{equation}
      \symup{A}_\text{g} = \frac{1}{2}\cdot3\cdot5=\frac{15}{2}=7.5
\end{equation}

\section*{h(x)}
Y-Achsenschnittpunkt: $S_y (0\,|-2)$
\begin{equation}
      h(0)=2\cdot 0-2=-2
\end{equation}
X-Achsenschnittpunkt: $S_x (1\,|\,0)$
\begin{align}
      0&=2x-2\;\;\;\;|\!+\!2\\
      2&=2x\;\;\;\;\;\;\;\;\;\;|\!:\!2\\
      x&=1
\end{align}
Flächeninhalt:
\begin{equation}
      \symup{A}_\text{f} = \frac{1}{2}\cdot1\cdot2=1 
\end{equation}

\section*{Schnittpunkte}
f und g:
\begin{align}
    -5x+1&=-\frac{3}{5}x+3 &\quad& |+\frac{3}{5}x \\
      -\frac{22}{5}x+1&=3 && |-1 \\
      -\frac{22}{5}x&=2 && | \cdot\!(-1)\\
      \frac{22}{5}x&=-2 && | \cdot\!\frac{5}{22}\\
      x&=-\frac{5}{11} \\\\
      f\left(-\frac{5}{11}\right)&=-5\cdot\left(-\frac{5}{11}\right)+1=\frac{25}{11}+1=\frac{36}{11}
\end{align}
$\symup{S}_\text{f,g}\left(-\frac{5}{11}\,\middle|\,\frac{36}{11}\right)$\\\\
f und h:
\begin{align}
    -5x+1&=2x-2 &\quad& | +5x\\
      1&=7x-2 && | +2\\
      3&=7x && | :7\\
      \frac{3}{7}&=x\\\\
      f\left(\frac{3}{7}\right)&=-5\cdot\frac{3}{7}+1=-\frac{15}{7}+1=-\frac{8}{7}
\end{align}
$\symup{S}_\text{f,h}\left(\frac{3}{7}\,\middle|\,-\frac{8}{7}\right)$\\\\
g und h:
\begin{align}
    -\frac{3}{5}x+3&=2x-2&\quad& | +\frac{3}{5}x\\
      3&=\frac{13}{5}x-2 && | +2\\
      5&=\frac{13}{5}x && | \cdot\!\frac{5}{13}\\
      \frac{25}{13}&=x \\\\
      g\left(\frac{25}{13}\right)&=-\frac{3}{5}\cdot\frac{25}{13}+3=-\frac{15}{13}+3=-\frac{2}{13}
\end{align}
$\symup{S}_\text{f,g}\left(-\frac{5}{11}\,\middle|\,\frac{36}{11}\right)$

\end{document}
