\documentclass[
  bibliography=totoc,
  captions=tableheading,
  titlepage=firstiscover,
]{scrartcl}
\usepackage{scrhack}
\usepackage[aux]{rerunfilecheck}
\usepackage{amsmath}
\usepackage{amssymb}
\usepackage{mathtools}
\usepackage{fontspec}
\recalctypearea{}
\usepackage{polyglossia}
\setmainlanguage{german}
\usepackage[
  math-style=ISO,
  bold-style=ISO,
  sans-style=italic,
  nabla=upright,
  partial=upright,
  warnings-off={
    mathtools-colon,
    mathtools-overbracket,
  },
]{unicode-math}
\setmathfont{Latin Modern Math}
\setmathfont{XITS Math}[range={scr, bfscr}]
\setmathfont{XITS Math}[range={cal, bfcal}, StylisticSet=1]
\usepackage[
  locale=DE,
  separate-uncertainty=true,
  per-mode=symbol-or-fraction,
]{siunitx}
\usepackage[
  version=4,
  math-greek=default,
  text-greek=default,
]{mhchem}
\usepackage[autostyle]{csquotes}
\usepackage{xfrac}
\usepackage{float}
\floatplacement{figure}{htbp}
\floatplacement{table}{htbp}
\usepackage[
  section,
  below,
]{placeins}
\usepackage{pdflscape}
\usepackage[
  labelfont=bf,
  font=small,
  width=0.9\textwidth,
]{caption}
\usepackage{subcaption}
\usepackage{graphicx}
\usepackage{grffile}
\usepackage{booktabs}
\usepackage{microtype}
\usepackage[
  unicode,
  pdfusetitle,
  pdfcreator={},
  pdfproducer={},
]{hyperref}
\usepackage{bookmark}
\usepackage[shortcuts]{extdash}


\begin{document}

\section*{Stauentwicklung auf der Autobahn}

An einer Autobahnbaustelle wird die Stauentwicklung von 08$:$00 Uhr bis 12$:$00 Uhr durch die Funktion
\begin{equation*}
    f(t) = t^3-6\cdot t^2+8\cdot t
\end{equation*}
dargestellt.
Dabei stellt $f(t)$ die Entwicklung des Staus in $\si{\kilo\meter\per\hour}$ dar und $t$ wird in $\si{\hour}$ angegeben.

\begin{itemize}
    \item Skizziere den Graphen von $f(t)$.
    \item Berechne die Nullstellen von $f$ und erkläre die Bedeutung dieser.
    \item Erkläre die Bedeutung positiver und negativer Funktionswerte von $f$.
    \item Bestimme rechnerisch die Zeitpunkte, zu denen die Staulänge am schnellsten zu- bzw. abnimmt.
        \begin{enumerate}
            \item Begründe, warum die Funktion $F$ mit
            \begin{equation}
                  F(t) = t^2\cdot\left(\frac{t^2}{4}-2t+4\right), \;\;\; t \in \left[0,4\right],
            \end{equation}
            die Staulänge zum Zeitpunkt $t$ beschreibt.
            \item Berechne die Staulänge für 08$:$30 Uhr.
            \item Berechne, um wie viel die Staulänge von 08$:$30 Uhr bis 09$:$00 Uhr zunimmt, und gib für diesen Zeitraum die durchschnittliche Änderungsrate der Staulänge an.
            \item Bestimme den Zeitpunkt, zu dem die Staulänge ihr Maximum erreicht hat, und berechne die Staulänge.
        \end{enumerate}
    \item An einem bestimmten Tag beginnt der Stau schon um 06$:$00 Uhr und hat sich bis um 10$:$00 Uhr vollständig aufgelöst.
        \begin{enumerate}
            \item Begründe, warum es nicht möglich ist, dass die momentane Änderungsrate der Staulänge durch die diffenrenzierbare Funktion $g$ modelliert werden kann.
                Der Graph dazu ist in \ref{fig:g} dargestellt.
            \item Ermittel die notwendige Bedingung, die jede differenzierbare Funktion $h$ erfüllen muss, damit diese die momentane Änderungsrate der Staulänge an einem Tag sinnvoll modellieren kann.
        \end{enumerate}
\end{itemize}
\begin{figure}
    \centering
    \includegraphics[width=0.9\textwidth]{build/plot2.pdf}
    \caption{Funktion $g$}
    \label{fig:g}
\end{figure}

\end{document}
