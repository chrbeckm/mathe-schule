\documentclass[
  bibliography=totoc,
  captions=tableheading,
  titlepage=firstiscover,
]{scrartcl}
\usepackage{scrhack}
\usepackage[aux]{rerunfilecheck}
\usepackage{amsmath}
\usepackage{amssymb}
\usepackage{mathtools}
\usepackage{fontspec}
\recalctypearea{}
\usepackage{polyglossia}
\setmainlanguage{german}
\usepackage[
  math-style=ISO,
  bold-style=ISO,
  sans-style=italic,
  nabla=upright,
  partial=upright,
  warnings-off={
    mathtools-colon,
    mathtools-overbracket,
  },
]{unicode-math}
\setmathfont{Latin Modern Math}
\setmathfont{XITS Math}[range={scr, bfscr}]
\setmathfont{XITS Math}[range={cal, bfcal}, StylisticSet=1]
\usepackage[
  locale=DE,
  separate-uncertainty=true,
  per-mode=symbol-or-fraction,
]{siunitx}
\sisetup{math-micro=\text{µ},text-micro=µ}
\usepackage[
  version=4,
  math-greek=default,
  text-greek=default,
]{mhchem}
\usepackage[autostyle]{csquotes}
\usepackage{xfrac}
\usepackage{float}
\floatplacement{figure}{htbp}
\floatplacement{table}{htbp}
\RequirePackage{luatex85}
\usepackage[
  locale=DE,
]{siunitx}
\usepackage{tikz}
\usepackage[
  europeanresistors,
  americaninductors,
  siunitx,
]{circuitikz}
\usepackage{stackengine}
\AtBeginDocument{
  \sisetup{
    math-rm=\mathrm,
    math-micro=µ,
  }
}
\usepackage[
  section,
  below,
]{placeins}
\usepackage{pdflscape}
\usepackage[
  labelfont=bf,
  font=small,
  width=0.9\textwidth,
]{caption}
\usepackage{subcaption}
\usepackage{graphicx}
% \usepackage{grffile} % currently broken
\usepackage{booktabs}
\usepackage{microtype}
\usepackage[
  backend=biber,
]{biblatex}
\addbibresource{lit.bib}
\addbibresource{programme.bib}
\usepackage[
  unicode,
  pdfusetitle,
  pdfcreator={},
  pdfproducer={},
]{hyperref}
\usepackage{bookmark}
\usepackage[shortcuts]{extdash}
\usepackage{multirow}
\usepackage{rotating}
\setlength\parindent{0pt}


\begin{document}

\section*{Stauentwicklung auf der Autobahn - Lösung}

An einer Autobahnbaustelle wird die Stauentwicklung von 08$:$00 Uhr bis 12$:$00 Uhr durch die Funktion
\begin{equation*}
    f(t) = t^3-6\cdot t^2+8\cdot t
\end{equation*}
dargestellt.
Dabei stellt $f(t)$ die Entwicklung des Staus in $\si{\kilo\meter\per\hour}$ dar und $t$ wird in $\si{\hour}$ angegeben.

\begin{itemize}
    \item Skizziere den Graphen von $f(t)$.
        \begin{figure}[H]
            \centering
            \includegraphics[width=0.7\textwidth]{build/plot.pdf}
            \caption{Skizze $f(t)$}
            \label{fig:f(t)}
        \end{figure}
    \item Berechne die Nullstellen von $f$ und erkläre die Bedeutung dieser.
      \begin{align*}
        &&f(t)=0 && | \text{Faktorisierung}\\
        \Leftrightarrow && t(t^2-6\cdot t + 8) = 0 \\
        \Rightarrow && t=0 \;\; \lor \;\; t^2-6\cdot t + 8 = 0 && | \text{pq-Formel}\\
        \Leftrightarrow && t_{\sfrac{1}{2}}=-\frac{-6}{2} \pm \sqrt{9-8} &&\\
        \Rightarrow && t=0, \;\; t=2, \;\; t=4.&&
      \end{align*}
      An den Nullstellen nimmt die Staulänge weder zu noch ab. Hier ist die Veränderung der Länge Null.\\
    \item Erkläre die Bedeutung positiver und negativer Funktionswerte von $f$.\\\\
      Während bei den Nullstellen die Staulänge unverändert bleibt, nimmt sie bei positiven Funktionswerten zu, und bei negativen Funktionswerten ab.\\
      Hierbei ist der Betrag des Funktionswerts die Geschwindigkeit, mit der sich die Staulänge verändert.\\
    \item Bestimme rechnerisch die Zeitpunkte, zu denen die Staulänge am schnellsten zu- bzw. abnimmt.\\\\
      Gesuchte Punkte: Hoch-/Tiefpunkte\\
      \begin{align*}
        \text{H.B.:} \;\; f'(t)=0&\\
        \text{N.B.:} \;\; f''(t)=0.&\\
        \Rightarrow & \;\; f'(t)=3\cdot t^2 - 12\cdot t + 8=0 \\
        \Leftrightarrow & \;\; 3\cdot t^2 - 12\cdot t + 8 = 0 &&|:3\; |\text{pq-Formel}\\
        \Leftrightarrow & \;\; t_{\sfrac{1}{2}}= -\frac{-4}{2} \pm \sqrt{4-\frac{8}{3}} = 2 \pm \frac{2}{\sqrt{3}}&&\\
        \Rightarrow &\;\; t=2-\frac{2}{\sqrt{3}} ,\;\; t=2+\frac{2}{\sqrt{3}}&&\\
        f''(t) = 6\cdot t - 12 &&\\
        \Rightarrow & \;\; f''\left(2-\frac{2}{\sqrt{3}}\right)=6\cdot\left(2-\frac{2}{\sqrt{3}}\right)-12 &&\\
        &= 12-\frac{12}{\sqrt{3}}-12 = -\frac{12}{\sqrt{3}} \;\; < 0 &&\\
        \Rightarrow & \;\; \text{Es handelt sich um einen Hochpunkt bei} \; t=2-\frac{2}{\sqrt{3}}&&\\
        \text{Bzw.} & &&\\
        & f''\left(2+\frac{2}{\sqrt{3}}\right)=6\cdot\left(2+\frac{2}{\sqrt{3}}\right)-12 &&\\
        &= 12+\frac{12}{\sqrt{3}}-12 = +\frac{12}{\sqrt{3}} \;\; > 0 &&\\
        \Rightarrow & \;\; \text{Es handelt sich um einen Tiefpunkt bei} \; t=2+\frac{2}{\sqrt{3}}.
      \end{align*}
      \begin{enumerate}
             \item Begründe, warum die Funktion $F$ mit
             \begin{equation*}
                   F(t) = t^2\cdot\left(\frac{t^2}{4}-2t+4\right), \;\;\; t \in \left[0,4\right],
             \end{equation*}
             die Staulänge zum Zeitpunkt $t$ beschreibt.\\\\
             Hier gibt es zwei verschiedene Wege:\\
             Entweder die gegebene Funktion $F(t)$ ableiten:\\
             \begin{align*}
               &&\left(F(t)\right)' &= \left(t^2\cdot\left(\frac{t^2}{4}-2t+4\right)\right)' \\
               \Leftrightarrow && f(t) &= \left(\frac{t^4}{4} - 2\cdot t^3 + 4\cdot t^2\right)'\\
               && &= \frac{4}{4}\cdot t^3 - 2\cdot 3\cdot t^2 + 4\cdot2\cdot t\\
               && &= t^3 - 6\cdot t^2 +8\cdot t.
               \intertext{oder die Funktion $f(t)$ unbestimmt integrieren:}
               &&\int_{}^{}{f(t)\;\symup{d}t} &= \int{t^3-6\cdot t^2+8\cdot t\;\symup{d}t}\\
               \Leftrightarrow && F(t) &= \frac{1}{4}t^4 - \frac{6}{3}t^3+\frac{8}{2}t^2\\
               && &= t^2\left(\frac{t^2}{4} - 2\cdot t + 4\right). &&\text{q.e.d.}
             \end{align*}
             \item Berechne die Staulänge für 08$:$30 Uhr.\\\\
               Die Staulänge an einem Zeitpunkt wird durch den Funktionswert der Funktion $F(t)$ beschrieben. Daher wird der Wert $F(0,5)$ berechnet:\\
               \begin{align*}
                 &&\text{08$:$00} &\cong 0\\
                 &&\text{12$:$00} &\cong 4\\
                 \Rightarrow && \text{08$:$30} &\cong 0,5.\\
                 \intertext{Damit kann die Staulänge bestimmt werden:}
                 && F(0,5) &= (0,5)^2\cdot \left(\frac{(0,5)^2}{4} - 2\cdot 0,5 + 4\right)\\
                 && &= \frac{1}{4}\left(\frac{1}{16}-1+4\right)\\
                 && &= \frac{49}{64}.
               \end{align*}
             \item Berechne, um wie viel die Staulänge von 08$:$30 Uhr bis 09$:$00 Uhr zunimmt, und gib für diesen Zeitraum die durchschnittliche Änderungsrate der Staulänge an.\\\\
               Eine Zunahme der Staulänge wird durch die Differenz zweier Funktionswerte bei den jeweiligen Zeitpunkten beschrieben.
               Man kann sich das vorstellen, wie das Integrieren von einem Zeitpunkt bis zum nächsten.
               Dieser Lösungsweg kann auch gewählt werden.
               Allerdings wird dieser hier nicht weiter ausgeführt, da die Integration oben bereits behandelt wurde.\\
               \begin{align*}
                 &&F(1)-F(0,5) &= 1\cdot \left(\frac{1}{4}-2+4\right)-\frac{49}{64}\\
                 && &= \frac{9}{4}-\frac{49}{64}\\
                 && &= \frac{144-49}{64} = \frac{95}{64}.
                 \intertext{Die Durchschnittliche Änderungsrate kann entweder mit der Formel für den Durchschnitt mit der Integration nach}
                 && \bar{m} &= \frac{1}{b-a}\int_{a}^{b}{f(t)\;\symup{d}t}
                 \intertext{berechnet werden, oder mit dem Ergebnis von oben und der Zeitdifferenz linear berechnet werden:}
                 &&\bar{m} &= \frac{95}{64}\cdot \frac{1}{2}\\
                 && &= \frac{95}{128}.
               \end{align*}
               Der Faktor $\sfrac{1}{2}$ stammt dabei aus der Zeitdifferenz von einer halben Stunde.
             \item Bestimme den Zeitpunkt, zu dem die Staulänge ihr Maximum erreicht hat, und berechne die Staulänge.\\\\
               Hier kann die Aufgabenstellung mit zwei Varianten gelöst werden.\\
               Entweder man berechnet das Maximum der Funktion $F(t)$,
               oder man überlegt sich aus dem Verlauf der Funktion $f(t)$,
               dass ab dem Zeitpunkt 10$:$00 Uhr die Staulänge wegen den Eigenschaften von $f(t)$ wieder abnimmt.
               Dann kommt man auf:\\
               \begin{align*}
                 && \left(F(t)\right)' &= f(t).\\
                 \Rightarrow && \text{Die Extrema entprechen der Nullstellen von $f(t)$.}\\
                 \Leftrightarrow && t=0,\;t=2,\;t=4.\\
                 \intertext{Da die Ableitung von $f(t)$, also $f'(t)$ nun entscheidet, ob es sich um ein Maximum oder Minimum handelt, müssen diese Eigenschaften für die Extremstellen überprüft werden.}
                 && f'(0) &= 3\cdot 0^2 - 12\cdot 0 +8 = 8\;>0\\
                 \Rightarrow && \text{Es handelt sich um einen Tiefpunkt.}\\
                 && f'(2) &=
         \end{enumerate}
     \item An einem bestimmten Tag beginnt der Stau schon um 06$:$00 Uhr und hat sich bis um 10$:$00 Uhr vollständig aufgelöst.
         \begin{enumerate}
             \item Begründe, warum es nicht möglich ist, dass die momentane Änderungsrate der Staulänge durch die diffenrenzierbare Funktion $g$ modelliert werden kann.
                 Der Graph dazu ist in \ref{fig:g} dargestellt.
            \item Ermittel die notwendige Bedingung, die jede differenzierbare Funktion $h$ erfüllen muss, damit diese die momentane Änderungsrate der Staulänge an einem Tag sinnvoll modellieren kann.
        \end{enumerate}
\end{itemize}
\begin{figure}
    \centering
    \includegraphics[width=0.7\textwidth]{build/plot2.pdf}
    \caption{Funktion $g$}
    \label{fig:g}
\end{figure}

\end{document}
