\documentclass[
  bibliography=totoc,
  captions=tableheading,
  titlepage=firstiscover,
]{scrartcl}
\usepackage{scrhack}
\usepackage[aux]{rerunfilecheck}
\usepackage{amsmath}
\usepackage{amssymb}
\usepackage{mathtools}
\usepackage{fontspec}
\recalctypearea{}
\usepackage{polyglossia}
\setmainlanguage{german}
\usepackage[
  math-style=ISO,
  bold-style=ISO,
  sans-style=italic,
  nabla=upright,
  partial=upright,
  warnings-off={
    mathtools-colon,
    mathtools-overbracket,
  },
]{unicode-math}
\setmathfont{Latin Modern Math}
\setmathfont{XITS Math}[range={scr, bfscr}]
\setmathfont{XITS Math}[range={cal, bfcal}, StylisticSet=1]
\usepackage[
  locale=DE,
  separate-uncertainty=true,
  per-mode=symbol-or-fraction,
]{siunitx}
\sisetup{math-micro=\text{µ},text-micro=µ}
\usepackage[
  version=4,
  math-greek=default,
  text-greek=default,
]{mhchem}
\usepackage[autostyle]{csquotes}
\usepackage{xfrac}
\usepackage{float}
\floatplacement{figure}{htbp}
\floatplacement{table}{htbp}
\RequirePackage{luatex85}
\usepackage[
  locale=DE,
]{siunitx}
\usepackage{tikz}
\usepackage[
  europeanresistors,
  americaninductors,
  siunitx,
]{circuitikz}
\usepackage{stackengine}
\AtBeginDocument{
  \sisetup{
    math-rm=\mathrm,
    math-micro=µ,
  }
}
\usepackage[
  section,
  below,
]{placeins}
\usepackage{pdflscape}
\usepackage[
  labelfont=bf,
  font=small,
  width=0.9\textwidth,
]{caption}
\usepackage{subcaption}
\usepackage{graphicx}
% \usepackage{grffile} % currently broken
\usepackage{booktabs}
\usepackage{microtype}
\usepackage[
  backend=biber,
]{biblatex}
\addbibresource{lit.bib}
\addbibresource{programme.bib}
\usepackage[
  unicode,
  pdfusetitle,
  pdfcreator={},
  pdfproducer={},
]{hyperref}
\usepackage{bookmark}
\usepackage[shortcuts]{extdash}
\usepackage{multirow}
\usepackage{rotating}
\setlength\parindent{0pt}


\begin{document}
    \section*{Kreise und Kreisringe}

    \subsection*{Formeln}
    \begin{table}
        \caption{Vervollständige die Tabelle. Verwende die allgemeinen Formeln}
        \label{tab:allgFormeln}
        \begin{tabular}{c | c c | c}
            \toprule
            & \multicolumn{2}{c}{Kreise}
            & {Kreisringe} \\
            & {Radius $r$}
            & {Durchmesser $d$}
            & {Innenradius $r_i$ \& Außenradius $r_a$}\\
            \midrule
            Fläche $A$ & $π \cdot r^2$ & $π \cdot \frac{d^2}{4}$ & $π \cdot (r_a^2 - r_i^2)$     \\
            & & & \\
            & & & \\
            Umfang $U$ & $2 \cdot π r$ & $π \cdot d$             & $2 \cdot π \cdot (r_a + r_i)$ \\
            & & & \\
            & & & \\
            \bottomrule
        \end{tabular}
    \end{table}

    \subsection*{Kreise}
    Gebe jeweils den Umfang und den Flächeninhalt des Kreises an.

    Für eine hilfsmittelfreie Lösung kann $π = \sfrac{22}{7}$ verwendet werden.
    \begin{align}
        r &= 5 &U = 10π &\approx \frac{220}{7} \approx 31,4 &A = 25π \approx \frac{550}{7} \approx 78,6 \\
        d &= 8 &U =  8π &\approx \frac{176}{7} \approx 25,1 &A = 16π \approx \frac{352}{7} \approx 50,3
    \end{align}
    ~\\
    Wie viele $\SI{0.8}{\meter}$ große Schritte muss man gehen um die Erde am Äquator zu umlaufen? \\
    Wie ist der Flächeninhalt der so umgangenen Erdscheibe? \\
    $d_\text{Erde} = \SI{6378137}{\meter}$
    \begin{align}
        U &= \frac{140319014}{7} \si{\meter} = \SI{20045573.4}{\meter} \\
        \# \text{Schritte} &=  \frac{\SI{20045573.4}{\meter}}{\SI{0.8}{\meter}} = \num{25056966.7} \\
        A &= \frac{22}{7} \cdot \frac{\SI{40680631590769}{\meter\squared}}{4} = \SI{31963353392747.1}{\meter\squared}
    \end{align}
    ~\\
    Mit eine der größten Uhren der Welt befindet sich in Mekka mit einer
    Zifferblattfläche von $\SI{43}{\meter\squared}$. Reicht dies für einen
    $\SI{3.5}{\meter}$ langen Zeiger? Wie groß müsste das Ziffernblatt sein,
    wenn der Abstand zwischen Zeigerende und Ziffernblattrand $\SI{1}{\meter}$ betragen soll.
    \begin{align}
        A_{\SI{3.5}{\meter}} &= \frac{22}{7} \cdot \left(\frac{7}{2} \si{\meter}\right)^2 = \frac{77}{2}\si{\meter} \approx \SI{38.5}{\meter} \\
        A_{\SI{4.5}{\meter}} &= \frac{22}{7} \cdot \left(\frac{9}{2} \si{\meter}\right)^2 = \frac{891}{14}\si{\meter} \approx \SI{63.6}{\meter}
    \end{align}

    \subsection*{Kreisringe}
    Die International Space Station befindet sich $\SI{370}{\kilo\meter}$ über der Erde,
    wie groß ist die Fläche die bei einer Umrundung zwischen Erde und ISS entsteht?
    $r_\text{Erde} \approx \SI{3189}{\kilo\meter}$
    \begin{align}
        A &= \frac{22}{7} \cdot \left( \left(\SI{3189}{\kilo\meter} + \SI{370}{\kilo\meter}\right)^2 - \left(\SI{3189}{\kilo\meter}\right)^2\right) \\
          &= \SI{7846960}{\kilo\meter\squared}
    \end{align}
    ~\\
    Ein Kreisel mit 3 Fahrspuren wird geplant. Jede Fahrspur soll $\SI{5}{\meter}$ breit sein.
    Wie viel Asphalt wird gebraucht, wenn der Kreisel einen Außenradius von $\SI{145}{\meter}$ haben soll?
    \begin{align}
        A &= \frac{22}{7} \cdot \left( \left(\SI{145}{\meter}\right)^2 - \left(\SI{145}{\meter} - 3\cdot\SI{5}{\meter}\right)^2\right) \\
          &= \SI{12964.3}{\meter\squared}
    \end{align}

    \end{document}
